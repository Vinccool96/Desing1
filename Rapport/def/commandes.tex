% !TeX encoding = UTF-8
% !TeX spellcheck = fr_FR


% Définition d'une commande pour avoir une ligne blanche entre deux paragraphes
\newcommand{\wl}{\par \vspace{\baselineskip}}

% Définitions de commandes pour faire des liens
\newcommand{\multiliens}[3]{
	\begin{scope}[x=1em,y=1em]
		\newdimen\xend
		\newdimen\yend
		\path (#2.west);
		\pgfgetlastxy{\xend}{\yend}
		\foreach \i in {#1} {
			\newdimen\xstart
			\newdimen\ystart
			\path (\i.east);
			\pgfgetlastxy{\xstart}{\ystart}
			\coordinate (1) at ({\xend-#3 em)},\ystart);
			\coordinate (2) at ({\xend-#3 em},\yend);
			\ifdim\ystart=\yend
				\draw[->] (\i.east)--(#2.west);
			\else
				\draw[->,rounded corners] (\i.east)--(1)--(2)--(#2.west);
			\fi
		}
	\end{scope}
}
\newcommand{\lienhorz}[2]{\draw[->] (#1.east) -- (#2.west);}
\newcommand{\lienvtht}[2]{\draw[->] (#1.north) -- (#2.south);}
\newcommand{\lienvtbs}[2]{\draw[->] (#1.south) -- (#2.north);}
\newcommand{\listebcs}[4]{
	\tikzset{
		barycentric setup/.code={\foreach \X [count=\Y] in {#1}
			{\ifnum\Y=1
				\xdef\baryarg{\X=1}
				\else
				\xdef\baryarg{\baryarg,\X=1}
				\fi}},
		barycentric list/.style={barycentric setup={#1},insert path={%
				(barycentric cs:\baryarg)}}
	}
	\path[barycentric list={#1}] node[anchor=center,align=flush center,#2] (#3) {#4};
}

\newcommand{\liensparfaits}[4]{%
	\begin{scope}[x=1em,y=1em]
		\listebcs{#2}{}{bcright}{}
		\newdimen\xright
		\newdimen\ybc
		\newdimen\dump
		\path(bcright);
		\pgfgetlastxy{\dump}{\ybc}
		\getpremier{#2}
		\path(\premier.west)--++(-#3em,0em);
		\newdimen\xmidright
		\pgfgetlastxy{\xmidright}{\dump}
		\coordinate (midright) at (\xmidright,\ybc);
		\path(midright)--++(-#4em,0em);
		\newdimen\xmidleft
		\pgfgetlastxy{\xmidleft}{\dump}
		\coordinate (midleft) at (\xmidleft,\ybc);
		\foreach \i in {#1} {
			\foreach \j in {#2}{
				\newdimen\ystart
				\path (\i.east);
				\pgfgetlastxy{\dump}{\ystart}
				\newdimen\xmidl
				\newdimen\xmidr
				\path (midleft);
				\pgfgetlastxy{\xmidl}{\dump}
				\path (midright);
				\pgfgetlastxy{\xmidr}{\dump}
				\newdimen\yend
				\path (\j.west);
				\pgfgetlastxy{\dump}{\yend}
				\coordinate (cl) at (\xmidl,\ystart);
				\coordinate (cr) at (\xmidr,\yend);
				\ifdim\ystart=\ybc\relax%
				\ifdim\ybc=\yend\relax%
				\draw[->] (\i.east)--(\j.west);%
				\else\relax%
				\draw[->,rounded corners] (\i.east)--(midright)--(cr)--(\j.west);%
				\fi\relax%
				\else\relax%
				\ifdim\ybc=\yend\relax%
				\draw[->,rounded corners] (\i.east)--(cl)--(midleft)--(\j.west);%
				\else\relax%
				\draw[->,rounded corners] (\i.east)--(cl)--(midleft)--(midright)--(cr)--(\j.west);%
				\fi\relax%
				\fi\relax%
			}
		}
	\end{scope}
}

\newcommand{\background}[3]{
	\begin{pgfonlayer}{background}
		\begin{scope}[x=1em,y=1em]
			\newdimen\dump
			\newdimen\xgauche
			\newdimen\yhaut
			\newdimen\xdroit
			\newdimen\ybas
			\path(#1.west)--++(-1.5em,0em);
			\pgfgetlastxy{\xgauche}{\dump}
			\path(#1.north)--++(0em,1.5em);
			\pgfgetlastxy{\dump}{\yhaut}
			\path(#2.east)--++(1.5em,0em);
			\pgfgetlastxy{\xdroit}{\dump}
			\path(#2.south)--++(0em,-1.5em);
			\pgfgetlastxy{\dump}{\ybas}
			\coordinate (a) at (\xgauche, \yhaut);
			\coordinate (b) at (\xdroit, \yhaut);
			\coordinate (c) at (\xdroit, \ybas);
			\coordinate (d) at (\xgauche, \ybas);
			\draw[-,rounded corners,color=red] (a)--(b)--(c)--(d)--cycle;
			\node[yshift=0.5em](center) at (barycentric cs:c=1,d=1) {#3};
		\end{scope}
	\end{pgfonlayer}
}

%manipulation string (pour liste)
\newcommand{\getpremier}[1]{
	\StrCount{#1}{,}[\nbrelems]
	\ifnum\nbrelems>0\relax
		\StrBefore[1]{#1}{,}[\premier]
	\else
		#1
	\fi
}