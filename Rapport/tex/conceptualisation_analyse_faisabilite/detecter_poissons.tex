\section{Prendre une photo}
	
	Pour la caméra, il est important, qu’elle ne consomme pas trop d’énergie, qu’elle soit plutôt compacte, car nous disposons d’un espace limité. Il faut aussi que la caméra puisse prendre des photos en au moins 16 pixels et 8 bits afin de pouvoir identifier l’espèce du poisson. Il est intéressant d’aussi tenir en compte des spécificités uniques aux différentes caméra, par exemple si celle-ci est déjà étanche.
	
	Aspects physiques:
	\begin{itemize}
		
		\item Il faut que la caméra prenne des photos en au moins 8 bits et 16 pixels.
		
	\end{itemize}
	
	Aspects économiques :
	
	\begin{itemize}
		
		\item Nous devons nous assurer que le prix est minimal.
		
	\end{itemize}
	
	Aspects temporels:
	
	\begin{itemize}
		\item La caméra doit avoir une autonomie suffisamment basse pour être considéré comme une option viable.		
		\item La livraison doit être faite dans un délais raisonnable.
	\end{itemize}

	Aspects socio-économiques:
	\begin{itemize}
	\item Non-applicable.
	\end{itemize}
	
	\subsection{Mini caméra Full HD 1080P DV sok3254}
	

	\textbf{Description}: Cette caméra à une volume de 1 pouce cube, ce qui n’est pas trop gros et permet de réduire la taille du modèle. La caméra possède une résolution de 1920 X 1080p, alors la qualité de l’image n’est pas un problème. La caméra permet une prise de photo à 150 degrés afin de maximiser l’information prise en une photo. Cette caméra a aussi un système de 6 LED intégré afin de prendre des photos la nuit ou en l'absence de lumières, ce qui est bénéfique pour faciliter le fonctionnement du système complet. Ensuite la caméra possède une mémoire de 32 gigaoctets, ce qui amplement suffisant pour la tâche demandée. Finalement, la caméra possède une autonomie de 7 jours, avec une batterie au lithium-ion de 200 milliampère par heure (mAh), ce qui est suffisant sachant que la caméra peut prendre des photos lors de la charge et que notre système comportera une batterie qui pourra recharger la caméra pour les 7 jours manquant. Le prix de cette caméra est de 14\$ canadien (normalement 50\$ canadien, mais sont en rabais) plus 8\$ canadien pour la livraison ainsi qu’un délais de 3 à 6 semaines. Cette solution coûterait à peu près 22\$ canadien avec le rabais et 58\$ canadien sans le rabais.
	
	\textbf{Décision}: retenue mais
	
	\textbf{Justification}: La caméra est conforme à toutes les demandes du client, et offre un système d’éclairage intégré.
	
	\textbf{Référence}: \cite{sok3254}
	
	\subsection{Papakoyal Mini Camera DCS-936L}

	
	\textbf{Description}: Cette caméra est capable de prendre des photos en HD 1080p ce qui est plus que suffisant pour le projet. Prend des photos à un angle de 140 degrés, ce qui est amplement pour le projet, cela permettra une plus grande prise d’information par photo. La batterie à une capacité de 200mAh et prend 2 à 3 heures pour recharger avec un voltage de 5V ceci devra être pris en compte lors de l’achat de la batterie, car l’autonomie de cette caméra est d’environ 24h. La caméra possède une mémoir de 16 gigaoctets, ce qui permet de s’assurer que s’il y a beaucoup de poissons, la caméra pourra stocker toutes les photos. Finalement, la caméra comporte des lumières LED qui permettent de prendre des photos dans le noir. Le coût de cette caméra est de 35\$ canadiens et la livraison est gratuite. La livraison peut se faire dans les 48h. La caméra pèse 68 grammes.

	\textbf{Décision}: Retenu mais

	\textbf{Justification}: Cette caméra répond aux critères demandés par le client, mais, n’apporte aucun point facilitant une autre partie du projet. Par exemple ça faible autonomie peut complexifier le rôle qu’on les batteries dans notre modèle.
	
	\textbf{Référence}: \cite{papakoyal963L}
	
	\subsection{Weatherproof TTL Serial JPEG Camera with NTSC Video and IR LEDs}

	\textbf{Description}: La caméra pèse 150 grammes et mesure 2 pouces par 2 pouces par 2,5 pouces. Ainsi la caméra n’est pas trop grosse et elle est assez légère. De plus, la caméra est étanche, ce qui permet une plus grande liberté dans le positionnement de celle-ci. La caméra fonctionne sur un voltage de 5 volts, pour un courant de 325 mA lorsque les LEDs sont allumées et 75 mA lorsqu’elle sont éteinte. La caméra comporte un système de détection de la lumière et ainsi n’allume que les LEDs intégrées que si la luminosité n’est pas assez bonne pour prendre la photo, ce qui permet de sauver beaucoup d’énergie. La caméra détecte avec un angle de 60 degrés. Les photos s’enregistre sous forme JPEG, qui est une forme très compacte de photos qui permet de prendre moins de mémoire et ainsi enregistrer plus de photos.
	
	\textbf{Décision}: Retenu
	
	\textbf{Justification}: La caméra répond à toutes les attentes du client ainsi qu’offre des solutions facilitantes pour les autre parties du projet, tel que les LED intégrés et le stockage en JPEG.
	
	\textbf{Référence}: \cite{camweatherproof}
	
	\subsection{Moosoo 1080P/720P HD 6 LED Infrared Night Vision Camera}

	\textbf{Description}:  La caméra permet de prendre des photos en 720p ce qui est amplement suffisant. La caméra comporte 4 LED qui permettent de photographier même en l’absence de lumières. La caméra est petite ce qui permet d’optimiser l’espace dans le boîtier. La caméra a une batterie de 200mAh et a une autonomie d’à peu près 100 minutes, ce qui n’est pas très bon pour notre design qui doit avoir une autonomie de 14 jours. La caméra pèse 90,7 grammes, ce qui permet de réduire le poids du système au maximum. Le prix de la caméra est de 26\$ canadiens, la livraison est au prix de 7\$ canadiens et est livrable dans les 72h.
	
	\textbf{Décision}: Retenue mais
	
	\textbf{Justification}: La caméra répond aux demandes du clients, mais n’apporte rien d’autre au projet que des LEDs intégré. De plus sa faible autonomie d’à peine 100 minutes complexifie grandement les autres composantes du modèle, par exemple les batteries.
	
	\textbf{Référence}: \cite{moosoocam}
	
\begin{table}[!htbp]
	\begin{tabular}{|l|c|c|c|c|c|}
		\hline
		\multicolumn{1}{|c|}{\multirow{2}{*}{\textbf{Concepts}}} & \multicolumn{4}{c|}{\textbf{Aspects de l'analyse}} & \multirow{2}{*}{\textbf{Décision}} \\ \cline{2-5}
		\multicolumn{1}{|c|}{}                                   & Physiques & Économiques & Temporels & Socio-envir. &                                    \\ \hline
		Mini caméra FullHD                                                  & OUI       & OUI         & OUI MAIS      & N/A          & RETENU MAIS                           \\ \hline
		DCS-936L                                                 & OUI       & OUI         & OUI MAIS       & N/A	& RETENU MAIS                             \\ \hline
		Weatherproof TTL Serial                                                & OUI       & OUI         & OUI MAIS       & N/A          & RETENU                             \\ \hline
		Moosoo 1080P/720P                                                 & NON       & OUI         & OUI MAIS       & N/A          & RETENU MAIS	        \\ \hline
	\end{tabular}
	\caption{Décisions pour fonction "Prendre une photo"}
	%\label{tab:fct_acceder}
\end{table}

	\subsection{Caisson étanche à l'eau}
	Afin d’assurer le bon fonctionnement de notre système sous l’eau, le client requiert que le système puisse tolérer des profondeurs d’au moins 15,25m. Le boitier qui couvrira les composantes, se devra donc d’être étanche même à 15,25m. De plus ce boitier ne devra pas être trop lourde, car le poids de la partie submergé est au maximum de 5 kilogrammes, il faudra donc tenir compte du poid des différents boitier proposés. Le boîtier doit aussi être assez grand, pour permettre de placer toutes les composantes du système à l’intérieur et que ceux-ci ne soient pas trop près les uns des autres afin de minimiser l’accumulation de chaleur et la surchauffe de notre système. Ainsi, nous avons fait fabriquer le boîtier sur mesure par une autre compagnie. Le boitier est composé de deux caissons, un sous l’eau et un sur l’eau, les deux reliés par un fil étanches qui permet d’acheminer le courant électrique et les informations prises aux cours de l’utilisation du boîtier. Ce boitier coute : 75\$. Le caisson submergé comprend une vitre transparente ainsi que des trous étanches afin de pouvoir y ajouter des accessoires, par exemple des sondes, sonar, thermomètre, etc. La conception du caisson coûte : 1000\$.
	
	\subsection{Sonar}
	Notre système comportera un sonar qui aura pour but de repérer les poissons pour que la caméra les prennent en photo. Ce sonar est le Module Sonar Imperméable JSN-B02\cite{sonarshop}, car il est peu coûteux (21,32\$ canadien), demande peu d’énergie( 5V et 30 mA pour fonctionner) et est compacte. De plus il est imperméable, ce qui permet plus d’option pour placer la sonde sur notre boitier.