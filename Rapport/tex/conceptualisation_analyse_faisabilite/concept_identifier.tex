% !TeX encoding = UTF-8
% !TeX spellcheck = fr_FR


\section{Identifier les espèces de poissons}
\label{s:faisab_identifier}
Le MFA exige pouvoir identifier 5 espèces de poissons identifiables par site étudié. L’identification doit être faite de manière autonome partir d’une vignette 100 par 100 pixels. Les concepts retenus doivent non seulement identifier les espèces mais offrir niveau de fiabilité acceptable. L’apprentissage automatique (de l’anglais “machine learning”) semble être la technologie à favoriser dans notre cas. Le choix de plateforme pour l’apprentissage automatique devrait minimiser le coût et le temps de développement.

Puisqu’il n’existe pas de banque d’image permettant d’entraîner la machine pour cette tâche, une telle collection devra être créée pour le projet. Environ 200 images identifiées par espèce devraient suffir pour entraîner l’algorithme.\cite{deeplearn_2015, fishID_2016} Faire classifier 200 images par un biologiste devrait prendre au plus 1 journée, pour un total de 240\$ \cite{sal_biologiste} par espèce. Les images à identifier manuellement proviendront des premières images générées par le projet. Accumuler 200 images d’une nouvelle espèce pourrait prendre jusqu’à 1 mois, mais s’effectue en parallèlement pour toutes les espèces d’un milieu particulier.

Aspects physiques:
\begin{itemize}
	\item L’identification doit pouvoir être fiable
	\item Doit pouvoir identifier au minimum 5 espèces de poisson.
\end{itemize}

Aspects économiques:
\begin{itemize}
	\item Limiter le coût de développement.
	\item Limiter le coût d’achat de matériel, si applicable.
\end{itemize}

Aspects temporels:
\begin{itemize}
	\item Le concept doit pouvoir être développé dans les délais du projet.
\end{itemize}

Aspects sociaux-environnementaux:
\begin{itemize}
	\item Non-applicable
\end{itemize}


% !TeX encoding = UTF-8
% !TeX spellcheck = fr_FR


\subsection{TensorFlow sur Google Cloud}
\label{s:identifier_conc1}

\textbf{Description}:TensorFlow est la plateforme “open source” d'apprentissage automatique de Google. L’utilisation de TensorFlow est gratuite en respectant la licence [ref:liscapache2]. Avec une banque d’image pré-identifiées, 3 semaines de travail sont estimée pour développer l'algorithme d’apprentissage et l’entraîner. Ceci coûtera environ 4615\$, selon le salaire moyen d’un ingénieur logiciel[ref:salingLog]. Dans ces conditions la fiabilité de l’identification devrait approcher 95\%[ref:fishID_2016]. Googl Cloud est un service nuagique qui permet l’apprentissage automatique. Faire fonctionner cet algorithme sur Google Cloud pendant 2 ans devrait coûter aux environs de 2200\$. Développer un banque d’images pour 5 espèces monte à 1200\$ et 1 semaine de travail. Incluant la banque d’image, ce concept nécessitera donc un total de 4 semaines de travail, 2200\$ en matériel et 5815\$.
\textbf{Décision}: Retenu
\textbf{Justification}: Les coût de développement et de matériel sont en raisonnables. Les ressources choisies sont appropriées pour obtenir une identification fiable d’un nombre approprié d’espèces. Un mois de développement est acceptable dans le cadre du projet.
\textbf{Références}: googCloud, tensorFlow
% !TeX encoding = UTF-8
% !TeX spellcheck = fr_FR


\subsection{Tensorflow sur Amazon SageMaker}
\label{s:identifier_conc2}

\textbf{Description}:Tel que mentionné plus haut [ref point1] TensorFlow est une plateforme d’apprentissage automatique libre d’utilisation.  Avec une banque d’image pré-identifiées, 3 semaines de travail sont estimées pour développer l'algorithme d’apprentissage et l’entraîner. Selon le salaire moyen d’un ingénieur logiciel, le coût est de 4615\$.  La fiabilité de l’identification devrait approcher 95\%. SageMaker est le service nuagique d’Amazon qui permet l’apprentissage automatique. Faire fonctionner cet algorithme sur SageMaker pendant 2 ans devrait coûter aux environs de 1900\$. Développer un banque d’images pour 5 espèces monte à 1200\$ et 1 semaine de travail. Ce concept nécessitera donc un total de 4 semaines de travail, 1900\$ en matériel et 5815\$ en développement.
\textbf{Décision}: Retenu
\textbf{Justification}: Les coût de développement et de matériel sont en raisonnables. Les ressources choisies sont appropriées pour obtenir une identification fiable d’un nombre approprié d’espèces. Un mois de développement est acceptable dans le cadre du projet.
\textbf{Références}:
\cite{tensorFlow, amzSage, fishID_2016, sal_ingLog}
% !TeX encoding = UTF-8
% !TeX spellcheck = fr_FR


\subsection{PyTorch sur Google Cloud}
\label{s:identifier_conc3}

\textbf{Description}:PyTorch est une autre plateforme d’apprentissage automatique libre d’utilisation.[ref:pyTorch]  Avec une banque d’image pré-identifiées, 3 semaines de travail sont estimée pour développer l'algorithme d’apprentissage et l’entraîner. Ceci coûtera environ 4615\$, selon le salaire moyen d’un ingénieur logiciel[ref:salingLog]. Dans ces conditions la fiabilité de l’identification devrait approcher 95\%[ref:fishID2016]. Faire fonctionner cet algorithme sur Google Cloud pendant 2 ans devrait coûter aux environs de 2200\$. Développer un banque d’images pour 5 espèces monte à 1200\$ et 1 semaine de travail. Incluant la banque d’image, ce concept nécessitera donc un total de 4 semaines de travail, 2200\$ en matériel et 5815\$.
\textbf{Décision}: Retenu
\textbf{Justification}: Les coût de développement et de matériel sont en raisonnables. Les ressources choisies sont appropriées pour obtenir une identification fiable d’un nombre approprié d’espèces. Un mois de développement est acceptable dans le cadre du projet.
\textbf{Références}: 	pytorch, googCloud
% !TeX encoding = UTF-8
% !TeX spellcheck = fr_FR


\subsection{Keras sur Google Cloud}
\label{s:identifier_conc4}

\textbf{Description}: Keras est une librairie libre d’usage qui facilite le développement de d’applications d’apprentissage automatique avec TensorFlow[ref:keras].  L’utilisation de TensorFlow est libre d’usage[ref:tensorFlow]. Avec une banque d’image pré-identifiées, 2 semaines de travail sont estimée pour développer l'algorithme d’apprentissage et l’entraîner. Ceci coûtera environ 3077\$, selon le salaire moyen d’un ingénieur logiciel[ref:sal_ingLog]. Dans ces conditions la fiabilité de l’identification devrait approcher 95\%[ref:fishID_2016]. Google Cloud est un service nuagique qui permet l’apprentissage automatique. Faire fonctionner cet algorithme sur Google Cloud pendant 2 ans devrait coûter aux environs de 2200\$. Développer un banque d’images pour 5 espèces monte à 1200\$ et 1 semaine de travail. Incluant la banque d’image, ce concept nécessitera donc un total de 4 semaines de travail, 2200\$ en matériel et 4276\$.
\textbf{Décision}: Retenu
\textbf{Justification}: Les coût de développement et de matériel sont en raisonnables. Les ressources choisies sont appropriées pour obtenir une identification fiable d’un nombre approprié d’espèces. Un mois de développement est acceptable dans le cadre du projet.
\textbf{Références}: keras, googCloud, tensorFlow, fishID_2016
% !TeX encoding = UTF-8
% !TeX spellcheck = fr_FR


\subsection{Tableau identifier}
\label{s:archiver_tableau}


