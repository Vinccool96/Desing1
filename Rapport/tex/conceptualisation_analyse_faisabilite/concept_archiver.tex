% !TeX encoding = UTF-8
% !TeX spellcheck = fr_FR


\section{Archiver les données}
\label{s:faisab_archiver}
Le MFA exige que l’ensemble des données soient conservées pour un minimum de deux ans. Les données sont constituées des éléments suivants: images originales, vignettes, identification, date et heure, température du système, température de l’eau et alarmes. La capacité en mémoire et la fiabilité doivent donc être suffisants pour garantir l’intégrité des données sur 2 ans.

Aspects physiques:
\begin{itemize}
	\item Le concept doit avoir assez de mémoire pour conserver 2 années d'information: 1 To.
	\item Le concept doit être assez fiable pour garantir l’intégrité des données sur 2 ans.
\end{itemize}

Aspects économiques:
\begin{itemize}
	\item Le concept doit minimiser le coût 
\end{itemize}

Aspects temporels:
\begin{itemize}
	\item Le concept doit pouvoir être construit et déployé dans les délais du projets.
\end{itemize}

Aspects sociaux-environnementaux:
\begin{itemize}
	\item Non-applicable
\end{itemize}



% !TeX encoding = UTF-8
% !TeX spellcheck = fr_FR


\subsection{Serveur PowerEdge T30}
\label{s:archiver_conc1}

\textbf{Description}:La compagnie Dell offre un serveur prêt à accepter jusqu’à 4 disques durs pour 604,80\$[refDell]. Ce serveur sera installé au MFA. Le montant inclut une garantie de 2 ans sur le matériel avec livraison le jour suivant. En ajoutant 2 disques durs Western Digital Red de 1 To pour 149,98\$[refAmz] branchés en RAID1 [refRAID], on a 1 To de mémoire redondante. Ces disques ont un temps moyen entre pannes de 1000000 h, c-à-d 114 années.  Le coût total du matériel se limite à 754,68\$. 
\textbf{Décision}: Retenu mais 
\textbf{Justification}: Ce matériel permet d’entreposer une quantité suffisante de données. Le matériel du serveur est garanti sur 2 ans. Les disques durs de 1 To ont une cote de fiabilité dépassant largement 2 ans, en plus d’être installés de manière redondante. Les données n’ont toutefois aucune redondance géographique, ce qui augmente le risque de perte de données. Toutes les composantes sont disponibles immédiatement. Le coût total est acceptable dans le contexte du projet.
\textbf{Références}: 
https://www.dell.com/en-ca/work/shop/cty/pdp/spd/poweredge-t30/pe\_t30\_12084
579,00\$ (03-17, dispo maintenant)
25,80\$ pour garantie pièces sur 2 ans, next-day deliv
https://www.amazon.ca/dp/B008JJLXO6/ref=twister\_B008VQ8IKY?\_encoding=UTF8\&psc=1
74,99\$ (03-17, dispo maintenant)
WD RED spec sheet
https://en.wikipedia.org/wiki/Standard\_RAID\_levels#RAID\_1

% !TeX encoding = UTF-8
% !TeX spellcheck = fr_FR


\subsection{Service nuagique Amazon S3}
\label{s:archiver_conc2}

\textbf{Description}:Amazon S3 est le service d'entreposage de données nuagique d’Amazon[refAmz]. Les données seront donc entreposées dans leur serveurs. Entreposer 1 To de données sur le service S3 Intelligent-Tiering coûtera un estimé 35\$ par mois, selon le niveau d’accès[refCalc](1To, 90\%, 50, 10000, 10000, 0, 10000, 1000).  Les frais d’entreposage des données sur 24 mois sont estimées à 840\$. Il en coûtera moins si on utilise moins d’espace. Amazon garantit la durabilité et l'accessibilité de nos données sur toute l’année.[refProt] Toutes les données sont entreposées de manière redondantes dans à différents lieux géographiques.
\textbf{Décision}: Retenu
\textbf{Justification}:  Le service Amazon S3 est élastique, donc on ne peut manquer de mémoire. L’intégrité de nos données est garantie à 99,999999999\%[refProt].L’espace d’entreposage peut être commandé à tout moment. Le prix est abordable dans le contexte du projet.
\textbf{Références}: 
refAmz, refAmzCalc, refAmzProt

% !TeX encoding = UTF-8
% !TeX spellcheck = fr_FR


\subsection{Service nuagique Microsoft Azure}
\label{s:archiver_conc3}

\textbf{Description}:Azure est le service nuagique de Microsoft. Les données seraient donc entreposer dans les serveurs de Microsoft. Incluant les frais d’accès, entreposer 1 To de données sur le service Azure Files coûtera 2250\$ pour 24 mois. Il en coûtera moins si on utilise moins d’espace. Il est possible d’obtenir plus de mémoire automatiquement, selon les besoins. Microsoft garantit l’intégrité des données à 99,999999999\%[ref:msLRS].
\textbf{Décision}: Retenu 
\textbf{Justification}: Le service Azure Files permet d’ajuster la mémoire utilisée donc nous en aurons toujours suffisamment. L’intégrité du stockage est garantie par Microsoft. Le prix est abordable dans le contexte du projet.
\textbf{Références}: 
msAzFilre, msRedunLRS

% !TeX encoding = UTF-8
% !TeX spellcheck = fr_FR


\subsection{Serveur dédié chez OVH}
\label{s:archiver_conc4}

\textbf{Description}: La compagnie OVH offre la location de serveurs d’entreposage. La mémoire du serveur consiste en 4 disques de 6 To en configuration RAID1. Les données sont donc protégées par redondance. La location du serveur monte à 135,99\$ par mois. La bande passante est incluse jusqu’à 1000 Mbps, ce qui est amplement suffisant pour tous les besoins du projet. Sur 24 mois, les frais s’élèvent à 3263,76\$.
 
\textbf{Décision}: Retenu mais

\textbf{Justification}: Ce concept offre beaucoup de mémoire pour l’entreposage, et la redondance des données sur plusieurs disque. L’absence de redondance géographique augmente le risque de perte de donnée. La commande peut être prête en 10 jours. Les frais de matériel sont acceptables mais élevés pour une seule section du projet.

\textbf{Références}: 
\cite{siteOVH}


% !TeX encoding = UTF-8
% !TeX spellcheck = fr_FR


\begin{table}[!htbp]
	\begin{tabular}{|l|c|c|c|c|c|}
		\hline
		\multicolumn{1}{|c|}{\multirow{2}{*}{\textbf{Concepts}}} & \multicolumn{4}{c|}{\textbf{Aspects de l'analyse}} & \multirow{2}{*}{\textbf{Décision}} \\ \cline{2-5}
		\multicolumn{1}{|c|}{}                                   & Physiques & Économiques & Temporels & Socio-envir. &                                    \\ \hline
		PowerEdge T30                                                 & OUI MAIS       & OUI         & OUI       & OUI          & RETENU MAIS                             \\ \hline
		Nuage Amazon S3                                                 & OUI       & OUI         & OUI       & OUI          & RETENU                             \\ \hline
		Nuage MS Azure                                                 & OUI       & OUI         & OUI       & OUI          & RETENU                             \\ \hline
		Serveur OVH                                                & OUI MAIS      & OUI MAIS        & OUI       & OUI          & RETENU MAIS        \\ \hline
	\end{tabular}
	\caption{Décisions pour fonction "archiver les données"}
	\label{tab:fct_archiver}
\end{table}
