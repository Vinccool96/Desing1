% !TeX encoding = UTF-8
% !TeX spellcheck = fr_FR


\subsection{TensorFlow sur Google Cloud}
\label{s:identifier_conc1}

\textbf{Description}:TensorFlow est la plateforme “open source” d'apprentissage automatique de Google. L’utilisation de TensorFlow est gratuite en respectant la licence [ref:liscapache2]. Avec une banque d’image pré-identifiées, 3 semaines de travail sont estimée pour développer l'algorithme d’apprentissage et l’entraîner. Ceci coûtera environ 4615\$, selon le salaire moyen d’un ingénieur logiciel[ref:salingLog]. Dans ces conditions la fiabilité de l’identification devrait approcher 95\%[ref:fishID2016]. Googl Cloud est un service nuagique qui permet l’apprentissage automatique. Faire fonctionner cet algorithme sur Google Cloud pendant 2 ans devrait coûter aux environs de 2200\$. Développer un banque d’images pour 5 espèces monte à 1200\$ et 1 semaine de travail. Incluant la banque d’image, ce concept nécessitera donc un total de 4 semaines de travail, 2200\$ en matériel et 5815\$.
\textbf{Décision}: Retenu
\textbf{Justification}: Les coût de développement et de matériel sont en raisonnables. Les ressources choisies sont appropriées pour obtenir une identification fiable d’un nombre approprié d’espèces. Un mois de développement est acceptable dans le cadre du projet.
\textbf{Références}: googCloud, tensorFlow