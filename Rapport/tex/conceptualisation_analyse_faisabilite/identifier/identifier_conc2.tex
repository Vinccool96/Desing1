% !TeX encoding = UTF-8
% !TeX spellcheck = fr_FR


\subsection{Tensorflow sur Amazon SageMaker}
\label{s:identifier_conc2}

\textbf{Description}:Tel que mentionné plus haut [ref point1] TensorFlow est une plateforme d’apprentissage automatique libre d’utilisation.  Avec une banque d’image pré-identifiées, 3 semaines de travail sont estimées pour développer l'algorithme d’apprentissage et l’entraîner. Selon le salaire moyen d’un ingénieur logiciel, le coût est de 4615\$.  La fiabilité de l’identification devrait approcher 95\%. SageMaker est le service nuagique d’Amazon qui permet l’apprentissage automatique. Faire fonctionner cet algorithme sur SageMaker pendant 2 ans devrait coûter aux environs de 1900\$. Développer un banque d’images pour 5 espèces monte à 1200\$ et 1 semaine de travail. Ce concept nécessitera donc un total de 4 semaines de travail, 1900\$ en matériel et 5815\$ en développement.
\textbf{Décision}: Retenu
\textbf{Justification}: Les coût de développement et de matériel sont en raisonnables. Les ressources choisies sont appropriées pour obtenir une identification fiable d’un nombre approprié d’espèces. Un mois de développement est acceptable dans le cadre du projet.
\textbf{Références}:
\cite{tensorFlow, amzSage, fishID_2016, sal_ingLog}