%!TEX encoding = UTF-8

%
% Chapitre "Description"
%

\chapter{Description}
\label{s:description}

L’équipe d’ingénieurs Sultan a été retenue pour concevoir le prototype d’un système autonome fixe pour le comptage et l’identification de la faune marine.
Ce système est conçu pour le Ministère de la Faune aquatique.
\wl
Il doit permettre de mesurer l’activité marine dans une masse d’eau, tout en compilant les données pour des statistiques et en offrant une fiabilité des données par rapport à la population marine.
Le système doit être en mesure de comptabiliser et identifier les espèces de poissons évoluant dans un écosystème aquatique.
Il doit aussi recueillir les images, archiver les données pour une validation ultérieure et assurer une interaction externe avec le prototype en cas de problèmes. Il peut être placé soit en milieu sauvage ou en milieu commercial.
Dans l’intérêt de diminuer les coûts de main-d’œuvre et de faciliter l’opération du système, ce dernier devra être à la fois fixe et autonome dans ces mesures diverses.

Le déploiement se fera en eau douce tempérée, soit les conditions normales des eaux au Québec.
Tous les spécimens analysés sont prédéterminés et le système sera conçu pour un volume d’eau spécifique.
Le capteur optique sera fixé par le ministère et doit être fonctionnel sous l’eau 24/7.
Les images prises par le capteur doivent être suffisamment petites pour pouvoir être traitées rapidement, mais assez précises pour identifier les poissons et elles seront toutes enregistrées pour un traitement ultérieur pendant une période d’au moins 2 ans.
Le système doit également pouvoir enregistrer les caractéristiques de l’environnement comme la température de l’eau et du système en plus de l’heure et de la date.
Le système doit générer des alarmes lorsqu’une composante est défectueuse ou ne fonctionne pas correctement.
Le capteur doit être configurable à distance grâce à une application, cette connexion doit être sécurisée.
\wl
Sultan proposera donc une solution performante, fiable et peu coûteuse qui permettra de satisfaire les consignes du client tout en occupant un espace minime et en n’affectant pas l’écosystème marin.


