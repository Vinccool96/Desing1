% !TeX encoding = UTF-8
% !TeX spellcheck = fr_FR


\subsection{Accès à distance}
\label{s:cdc_com_accadist}

Dans le but de rendre le système autonome et simplifier l’utilisation de celui-ci, une communication à distance doit être configurée.
Cette caractéristique est cruciale étant donné de la position du capteur, qui n’est pas facilement accessible, car il est placé à une certaine profondeur dans l’eau.
C’est pourquoi une pondération de 10~\% lui est associée. Deux éléments principaux constitueront l’évaluation de ce critère.
Premièrement, le système doit admettre une communication rapide à distance en tout temps.
Selon l’atteinte de ce critère, une valeur entre 0 et 1 leur sera transmise selon la réussite ou l’échec du critère.
La valeur est déterminé par la vitesse de la connexion selon cette formule où $x$ est la vitesse de connexion en $ko/s$:
\begin{equation} \label{eq:cdc_com_accadist}
	\ln \left( \frac{x-384}{6816} \right) +1,\qquad x \in [384,\,7200]
\end{equation}