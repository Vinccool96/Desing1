% !TeX encoding = UTF-8
% !TeX spellcheck = fr_FR


\subsection{Configurabilité}
\label{s:cdc_aut_config}

Un micrologiciel implanté à notre système lui permettra de gérer les liens entre les différents capteurs, l’alimentation et le système de conservation des données accumulées.
Ce micrologiciel a aussi pour but de permettre la configuration du capteur pendant et après les prises de données.
Avant la mise en place de notre système dans un milieu d’étude on pourra lui indiquer les trois paramètres relatifs à l’environnement: milieu sauvage ou de culture, grandeur approximative du volume d’eau, période hivernale ou estivale.
Une fois les paramètres indiqués au système, ce dernier sera en mesure d’agir d’une manière autonome.
Puis, durant la période de prise de données, le système se règlera automatiquement en fonction de la luminosité et de la température de l’eau.
\wl
Pour évaluer ce critère, une note entre 0 et 100 sera attribuée selon son adaptation aux trois configurations.
Pour chaque configuration à laquelle il s’adapte, la note augmente de 25.
\wl
Une note de 0 est inacceptable, 25 est médiocre, 50 est moyen.
La note 75 est acceptable et enfin 100 est optimale.
La pondération est de 7~\%.