% !TeX encoding = UTF-8
% !TeX spellcheck = fr_FR


\subsection{Contrainte physique}
\label{s:cdc_pdd_cntphy}

Le client n’a pas indiqué de proportions claires à respecter pour la taille de notre système, celui-ci devra tout de même être le plus compact possible tout en assurant qu’une analyse sur un volume d’au moins un mètre cube d’eau soit possible.
Le système doit tolérer des températures entre 4~°C et 25~°C.
Afin d’éviter les problèmes de surchauffe, le produit final doit aussi compter un système de refroidissement automatique afin que la température intérieure soit au plus bas 10~°C en dessous de la température de l’eau et au plus haut cinq~°C au-dessus.
Le client demande aussi que le système soit assez résistant pour fonctionner à 15,25~m sous l’eau.
De plus, la limite de la masse de la partie submergée est fixée à 5 kilogrammes.
Un choix judicieux de matériaux légers, mais robustes est donc de mise.
Nous pondérons ce critère à 5~\%. Pour s’assurer de bien respecter ce critère, un tableau est fourni:

\begin{table}[htp]
	\caption{Barème de contraintes physiques}
	\label{tab:cdc_pdd_cntphy}
	\centering
	\begin{tabular}{|l|c|c|c|}
		\hline
		\multirow{2}{*}{\textbf{Partie évaluée}} & \multicolumn{3}{c|}{\textbf{Condition pour avoir la note de}} \\\cline{2-4}
		 & 0,2 & 0,15 & 0 \\\hline
		Volume analysé $V$ (m\textsuperscript{3}) & $V \ge 1$ & $V=1$ & $V<1$  \\\hline
		Température supportée $T$ (°C) & $4 \le T \le 25$  & non attribuée & 0 \\\hline
		Température par rapport à l'eau $\Delta T$ (°C) & $-10 \le \Delta T \le 5$ & non attribuée & 0 \\\hline
		Profondeur maximale $p$ (m) & $p \ge 15,25$ & $p=15,25$ & $p<15,25$    \\\hline
		Masse submergée $m$ (kg) & $m \ge 5$ & & $m < 5$  \\\hline
	\end{tabular}
\end{table}