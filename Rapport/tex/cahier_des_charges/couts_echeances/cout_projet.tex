% !TeX encoding = UTF-8
% !TeX spellcheck = fr_FR


\subsection{Coût du projet}
\label{s:cdc_cee_coutproj}

Ce critère est subdivisé en deux sous-catégories soit les coûts matériels et les coûts de main-d’œuvre.
La partie matérielle englobe l’ensemble des pièces et des produits qui seront utilisés pour mettre au point le capteur ainsi que la plateforme automatisée.
Le MFA a mis à notre disposition une somme maximale de 10 000~\$ pour rassembler les pièces nécessaires à la construction du produit.
Enfin, pour la partie main-d’œuvre, qui représente principalement le salaire des employés travaillant sur le projet, un total de 40 000~\$ pourra être utilisé au maximum.
Le barème est présenté au tableau \ref{tab:cdc_cee_coutproj}.

\begin{table}[H]
	\renewcommand\arraystretch{1.5}
	\centering
	\begin{tabular}{|l|c|c|}
		\hline
		& Pondération & Coût (\$)         \\ \hline
		\multirow{5}{*}{Coûts matériels}       & 0           & $[10000,\,\infty[$  \\ \cline{2-3}
		& 1.25 & $[7500,\,10000[$ \\ \cline{2-3}
		& 2.5 & $[5000,\,7500[$ \\ \cline{2-3}
		& 3.75 & $[2500,\,5000[$ \\ \cline{2-3}
		& 5 & $[0,2500[$ \\ \hline
		\multirow{5}{*}{Coûts de main-d’œuvre} & 0           & $[40000,\,\infty[$  \\ \cline{2-3}
		& 1.25 & $[30000,\,40000[$ \\ \cline{2-3}
		& 2.5 & $[20000,\,30000[$ \\ \cline{2-3}
		& 3.75 & $[10000,\,20000[$ \\ \cline{2-3}
		& 5 & $[0,10000[$ \\ \hline
	\end{tabular}
	\caption{Barème pour le coût du projet}
	\label{tab:cdc_cee_coutproj}
\end{table}