%!TEX encoding = UTF-8

%
% Chapitre "Introduction"
%

\chapter{Introduction}
\label{s:intro}

La société fait face à de nombreux défis environnementaux. Le déclin des populations aquatiques est un de ses enjeux, d’envergure mondiale.
Dans ce contexte, il est vital de pouvoir étudier la faune et la flore.
Étudier les populations animales dans un milieu donné est essentiel à la compréhension des changements qui ont lieu.
C’est pourquoi des outils adaptés à cette tâche sont importants.
\wl
Ainsi, le Ministère de la Faune aquatique a contacté notre firme afin de concevoir un système permettant le comptage et l’identification de la faune marine dans un milieu donné, grâce à un système autonome fixe.

Actuellement, l’acquisition de donnée est manuelle.
Elle est donc coûteuse en main-d’œuvre et intrusive pour la faune.
Le projet du Ministère de la Faune aquatique vise à automatiser son système d’acquisition de données.
Ce capteur autonome doit être passif, donc sans perturbation pour la faune.
Les données acquises sont précieuses pour étudier un biome aquatique précis.
\wl
Ce projet se trouve au croisement de divers domaines, tant environnementaux qu’économiques.
Il a une utilité scientifique, commerciale, économique et s’inscrit parfaitement dans une optique de recherche sur l’environnement.
\wl
Notre firme doit dans un premier temps décrire et définir ce projet.
À partir de ces objectifs, le présent document comprend un cahier des charges, l’analyse de différents concepts et la présentation du concept final retenu.