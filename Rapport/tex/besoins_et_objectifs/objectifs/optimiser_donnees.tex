% !TeX encoding = UTF-8
% !TeX spellcheck = fr_FR


\subsection{Optimiser la prise de données}
\label{s:beo_obj_optimdonn}

Il faudra que notre modèle puisse analyser un volume d’intérêt d’au moins 1~m\textsuperscript{3} cube d’eau.
Il doit pouvoir identifier des poissons d’au moins 5 espèces différentes et d’au moins 6 cm de long.
Les conditions climatiques du Québec étant très variables, notre système doit tolérer une température externe entre 4~°C et 25~°C et une température interne entre -10 °C et 5 °C par rapport à la température de l’eau.
La masse totale de l’équipement submerger ne doit pas dépasser 5 kilogrammes pour un volume maximum de 0,3~m\textsuperscript{3}.
Bien que léger, notre système devra être assez robuste pour fonctionner jusqu’à 15,25~m.
Les photos prises devront au minimum être codées en 8 bits dans une taille fixe de 100 pixels sur 100 pixels.
Chaque image devra inclure : l’heure, la date, la température interne du système, la température de l’eau et l’identification du poisson.