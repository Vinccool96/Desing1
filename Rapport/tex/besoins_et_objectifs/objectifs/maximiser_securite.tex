% !TeX encoding = UTF-8
% !TeX spellcheck = fr_FR


\subsection{Maximiser la sécurité}
\label{s:beo_obj_optisecu}

L’accès à distance doit être sécurisé. Cet accès sera chiffré pour éviter toute intrusion externe dans notre système. Seul un opérateur du MFA pourra consulter les données stockées ou les alarmes. Les données stockées après l’étude de l’écosystème seront également protégées car ce sont des données sensibles et seul le MFA pourra les consulter.
Notre groupe est sensible aux idées actuelles à propos de la protection de l’environnement. Ainsi une part de la protection pour l’environnement sera effectué avec un système non polluant et respectueux de l’écosystème. 
Deux niveaux d’alarmes vont être mis en place. Le premier niveau est faible et les alarmes seront stockées textuellement au coeur de notre système. Ce niveau n’a pas la nécessité de contacter immédiatement un opérateur du système. Le deuxième niveau est fort et nécessite un contact immédiat avec l’opérateur système (extinction prématurée, capteur hors service, etc.), car le problème empêche la prise de mesure.
