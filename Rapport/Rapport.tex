% !TeX encoding = UTF-8
% !TeX spellcheck = fr_FR

%
% Exemple de rapport
% par Pierre Tremblay, Universite Laval
% modifié par Christian Gagne, Universite Laval
% 14/01/2011 - version 1.3
% modifié par Robert Bergevin, Université Laval
% 24/11/2011
% modifié par Jean-Yves Chouinard, Université Laval
% 11/01/2016
% modifié par Jean-Yves Chouinard, Université Laval
% 04/01/2017
%

%
% Modele d'organisation d'un projet LaTeX 
% rapport/      dossier racine et fichier principal
% rapport/fig   fichiers des figures
% rapport/tex   autres fichiers .tex
%

% ** Preambule **
%
% Ajouter les options au besoin :
%    - "ULlof" pour inclure la liste des figures, requis si "\begin{figure}" utilise
%    - "ULlot" pour inclure la liste des tableaux, requis si "\begin{table}" utilise
%
\documentclass[12pt,ULlof,ULlot]{ULrapport}

% Chargement des packages supplementaires (si absent de la classe)
\usepackage[utf8]{inputenc}
\usepackage[autolanguage]{numprint}
\usepackage{icomma}
\usepackage[edges]{forest}
\usepackage{multirow}
\usepackage{float}
\usepackage{tikz}
\usepackage{amsfonts}
\usepackage{bbding}
\usepackage{amsmath}
\usepackage{lmodern}
\usetikzlibrary{calc}
\usetikzlibrary{matrix}
\usetikzlibrary{fit}
\usetikzlibrary{trees}
\usetikzlibrary{arrows.meta}
\usetikzlibrary{positioning}
\usetikzlibrary{shapes}
\usetikzlibrary{backgrounds}
\restylefloat{table}
\hypersetup{citecolor=red}
% %% to load before \documentclass %%
% check for common mistakes and obsolete packages
%\RequirePackage[l2tabu,orthodox]{nag}

%% make this files behave like a package %%
%% include with \usepackage{ultipack} %%
%\ProvidesPackage{ultipack}


%% common packages %%
% desirable LaTeX 2e fixes
\usepackage{fixltx2e}
% typographic improvements
\usepackage{microtype}
% UTF-8 input encoding and T1 font encoding
\usepackage[utf8]{inputenc}
\usepackage[T1]{fontenc}
% Latin modern font families (for accented characters)
\usepackage{lmodern}
% extended conditional commands
\usepackage{xifthen}
% LaTeX internationalization
\usepackage[french, english]{babel}
% setup for french in babel package
\frenchbsetup{
    CompactItemize=false, % do not compact lists
    ThinSpaceInFrenchNumbers=true % thin space in numbers
}

%% layout and structure packages %%
% change the margins so that the full page is used
%\usepackage{fullpage}
% customize the page geometry
%\usepackage[letterpaper]{geometry}
%\geometry{options}

% change the line spacing
\usepackage{setspace}
% remove paragraph indentation
\usepackage[parfill]{parskip}
% control paragraph indentation
%\setlength{\parindent}{1cm}
% easily input or include files in folders
%\usepackage{import}

% fine control of pages headers and footers
%\usepackage{fancyhdr}
% contents of headers and footers
%\pagestyle{options}
% format of page numbers
%\pagenumbering{options}

% landscape environment (with PDF support)
%\usepackage{pdflscape}
% create multicolumns documents
%\usepackage{multicol}

% advanced quotes environments and control
\usepackage[autostyle]{csquotes}
% better control of appendices
%\usepackage[toc]{appendix}


%% mathematical packages %%
% load amsmath, correct bugs and provide useful tools
\usepackage{mathtools}
% allow bold math symbols (replaces amsbsy)
\usepackage{bm}
% AMS math font and symbols
\usepackage{amsfonts,amssymb}
% AMS theorem environment
\usepackage{amsthm}
% consistent typesetting of numbers and units
\usepackage{siunitx}
% setup for siunitx package
%\sisetup{
    %free-standing-units, % to use units abbreviations outside si commands
    %space-before-unit, % add a space between a number and a free unit
    %use-xspace % add a space after a free unit if necessary
%}


%% tables and figures packages %%
% extended graphics functionalities
\usepackage{graphicx}
% improved tables
\usepackage{booktabs,tabu,multirow}
% multipage tables
%\usepackage{longtable}
% new column type: centered paragraph
\newcolumntype{C}[1]{>{\centering\arraybackslash}p{#1}}
% new rule type for under the table header
\newcommand*{\otoprule}{\midrule[\heavyrulewidth]}
% better control of captions and subcaptions
\usepackage{caption,subcaption}
% setup for caption and subcaption packages
\captionsetup{
    format=plain, % format of the caption: normal paragraph
    justification=centerlast, % center the last lineof caption
    width=0.9\textwidth, % width of caption
    subrefformat=parens % add parentheses to subfig reference
}
% rename the french tables caption to "Tableau" instead of "Table"
\addto\captionsfrench{\renewcommand{\tablename}{\textsc{Tableau}}}
% control captions for multipages floats
%\usepackage{captcont}

% create or include Asymptote figures
%\usepackage{asymptote}
% define the location folder for generated Asymptote files
%\def\asydir{asy}
% create or include Tikz/PGF figures
%\def\pgfsysdriver{pgfsys-dvipdfm.def}
%\usepackage{tikz}
%\usepackage{pgfplots}
% compatibility mode for PGF plots
%\pgfplotsset{compat=newest}
% disable support for math expressions in plot coordinates
%\pgfplotsset{plot coordinates/math parser=false}
% list of Tikz/PGF libraries to load
%\usetikzlibrary{list,of,libraries}
% define new variables to set figure dimensions
%\newlength\figurewidth
%\newlength\figureheight

%% programming packages %%
% include program listings into the document
%\usepackage{listings}
% setup for Matlab listings
%\lstset{language=Matlab,
    %basicstyle=\footnotesize,
    %linewidth=0.95\textwidth,
    %frame=single,
    %tabsize=2,
    %breaklines=true,
    %numbers=right,
    %numberstyle=\tiny,
    %numbersep=5pt,
    %showspaces=false,
    %showtabs=false,
    %showstringspaces=false
%}
% fancy verbatim environments
%\usepackage{fancyvrb}
% produce algorithms/pseudo-code
%\usepackage{algpseudocode}


%% bibliography and cross-referencing packages %%
% biblatex with Biber backend and numeric-comp style
\usepackage[backend=biber,style=numeric-comp]{biblatex}
% setup for biblatex package
\ExecuteBibliographyOptions{
    sorting=none, % references are listed in citation order
    url=false, % print url when appropriate
    doi=false, % never print DOI
    isbn=false % never print ISBN/ISSN/ISRN
}
% create a list of acronyms, symbols, etc.
%\usepackage{acronym}
%\usepackage{nomentbl}


%% miscellaneous packages %%
% automatically add a space after symbols (loaded by other packages)
%\usepackage{xspace}

% colors manipulations (often loaded by other packages)
%\usepackage[table]{xcolor}
% generate a dummy document
%\usepackage{blindtext}
% add todo notes (add option "final" to disable)
%\usepackage[colorinlistoftodos]{todonotes}
% define commands for new types of todos
%\newcommand{\longtodo}[1]{\todo[inline]{#1}}
%\newcommand{\addref}{\todo[color=red!40]{Add reference(s).}}
%\newcommand{\addcont}[2][]{\ifthenelse{\equal{#1}{}}{\todo[inline,color=green!40,caption={#2}]{#2}}{\todo[inline,color=green!40,caption={#1}]{#2}}}
%\newcommand{\note}[2][]{\ifthenelse{\equal{#1}{}}{\todo[inline,color=blue!40,caption={#2}]{#2}}{\todo[inline,color=green!40,caption={#1}]{#2}}}
%\newcommand{\addcont}[1]{\todo[inline,color=green!40,caption={Add contents.}]{#1}}
%\newcommand{\note}[1]{\todo[inline,color=blue!40,caption={Note.}]{#1}}


%% must be loaded at the end %%
% control hyperlinks in the document
\usepackage{hyperref}
% setup for hyperref package
\hypersetup{
    colorlinks, % color links
    %allcolors=blue, % set the color for all links
    citecolor=green, % set the color for citations
    filecolor=cyan, % set the color for files links
    linkcolor=red, % set the color for document links
    urlcolor=blue % set to color for URLs
}
% improved cross-referencing
\usepackage[french,english,noabbrev]{cleveref}
% create glossaries, symbols, acronyms, etc. (instead of acronym or nomentbl)
%\usepackage[xindy,acronym,nonumberlist,nosuper,notree,shortcuts,nohypertypes=]{glossaries}
% print defined labels (add option "final" to disable)
%\usepackage[notref,notcite]{showkeys}

%% include a list of necessary packages and files in the log file %%
%\listfiles
% possible to combine with bundledoc to create an archive with all the files needed


%\usepackage[options]{nom_du_package}

% Definition d'une commande pour presenter des cellules multilignes dans un tableau
\newcommand{\cellulemultiligne}[1]{\begin{tabular}{@{}c@{}}#1\end{tabular}}

% Définition d'une commande pour avoir une ligne blanche entre deux paragraphes
\newcommand{\wl}{\par \vspace{\baselineskip}}

% Definition de colonnes en mode paragraphe avec alignement ajustable
% Cette definition requiert le chargement du package "array"
%    - alignement horizontal, parametre #1 : - \raggedright (aligne a gauche)
%                                            - \centering (centre)
%                                            - \raggedleft (aligne a droite)
%    - alignement vertical, parametre #2 : - p (aligne en haut)
%                                          - m (centre)
%                                          - b (aligne en bas)
%    - largeur, parametre #3 : longueur
\newcolumntype{Z}[3]{>{#1\hspace{0pt}\arraybackslash}#2{#3}}

% Definitions des parametres de la page titre
\TitreProjet{Projet Fish \& Chips}                         % Titre du projet
\TitreRapport{Rapport de projet -- version 0}                       % Titre du rapport
\Destinataire{Robert Bergevin, Luc Lamontagne et Simon Thibault}         % Nom(s) du destinataire
\NumeroEquipe{8}                                      % Numero de l'equipe
\NomEquipe{Sultan}                               % Nom de l'equipe
\TableauMembres{%                                     % Tableau des membres de l'equipe
	111\,233\,871  & Vincent Girard				& \\\hline        % matricule & nom & \\\hline
	906\,219\,729  & Luc Rainville				& \\\hline        % matricule & nom & \\\hline
	111\,226\,718  & Vincent Breault				& \\\hline        % matricule & nom & \\\hline
	111\,241\,694  & Thomas Gaillard				& \\\hline        % matricule & nom & \\\hline
	111\,240\,963  & Nicolas Cabanac Lassonde	& \\\hline        % matricule & nom & \\\hline
	111\,236\,540  & Hassan Bourezak				& \\\hline        % matricule & nom & \\\hline
}
\DateRemise{31 janvier 2019}                           % Date de remis


% Contenu de l'historique des versions
\HistoriqueVersions{
	%	# & date & description \\\hline
	0.1 & 31 janvier 2019 & création du document \\\hline
	1.1 & 21 février 2019 & Besoins, objectifs et cahier des charges \\\hline
}


% Corps du document

\begin{document}
	
	%   Chapitres
	%
	%!TEX encoding = IsoLatin

%
% Chapitre "Introduction"
%

\chapter{Introduction}
\label{s:intro}

Ici va le texte d'introduction



	%!TEX encoding = UTF-8

%
% Chapitre "Description"
%

\chapter{Description}
\label{s:description}

L’équipe d’ingénieurs Sultan a été retenue pour concevoir le prototype d’un système autonome fixe pour le comptage et l’identification de la faune marine.
Ce système est conçu pour le Ministère de la Faune aquatique.
\wl
Il doit permettre de mesurer l’activité marine dans un volume fixe d’eau, tout en compilant les données pour des statistiques et en offrant une fiabilité des données par rapport à la population marine.
Le système doit être en mesure de comptabiliser et identifier les espèces de poissons évoluant dans un écosystème aquatique.
Il doit aussi recueillir les images, archiver les données pour une validation ultérieure et assurer une interaction externe avec le prototype en cas de problèmes. Il peut être placé soit en milieu sauvage ou en milieu commercial.
Dans l’intérêt de diminuer les coûts de main-d’œuvre et de faciliter l’opération du système, ce dernier devra être à la fois fixe et autonome dans ses opérations diverses.
\wl
Le déploiement se fera en eau douce tempérée, soit les conditions normales des eaux au Québec.
Tous les spécimens analysés sont prédéterminés et le système sera conçu pour un volume d’eau spécifique.
Le tout sera installé sur le terrain par le ministère et doit être fonctionnel sous l’eau sans interruption.
Les images prises par le capteur doivent être suffisamment petites pour pouvoir être traitées rapidement, mais assez précises pour identifier les poissons. 
Elles seront toutes enregistrées pour un traitement ultérieur pendant une période d’au moins 2 ans.
Le système doit également pouvoir enregistrer les caractéristiques de l’environnement comme la température de l’eau et du système en plus de l’heure et de la date.
Le système doit générer des alarmes lorsqu’une composante est défectueuse ou ne fonctionne pas correctement.
Le capteur doit être configurable à distance grâce à une application et cette connexion doit être sécurisée.
\wl
Sultan propose donc dans ce document une solution performante, fiable et peu coûteuse qui répond aux exigences du Ministère de la Faune aquatique.



	% !TeX encoding = UTF-8
% !TeX spellcheck = fr_FR


\chapter{Besoins et objectifs}
\label{s:beo}

Ce chapitre présente les besoins et les objectifs que le MFA a énoncés lors de la description du mandat.
On y retrouve une description de chacun des besoins ainsi qu’une analyse de chaque objectif.

% !TeX encoding = UTF-8
% !TeX spellcheck = fr_FR


\section{Besoins}
\label{s:beo_bes}

% !TeX encoding = UTF-8
% !TeX spellcheck = fr_FR


\subsection{Accumulation de données}
\label{s:beo_bes_accumdonnees}
Le système de mesures doit pouvoir comptabiliser et identifier plusieurs espèces de poissons qui évoluent sur un site marin en eau douce.
Les données prennent la forme d’images couleur. Une vignette (portion d’image) contenant le sujet doit être sauvegardée pour chaque poisson identifié.
À cette vignette sont associées des informations sur le milieu d’étude au moment de la prise de données.
La qualité des mesures doit être de qualité constante.
\wl
Le capteur optique doit être contenu dans une enceinte étanche pouvant être ancrée à différentes profondeurs par le client.
Ce système doit pouvoir opérer en continu dans des conditions tempérées.
% !TeX encoding = UTF-8
% !TeX spellcheck = fr_FR


\subsection{Automatisation du système}
\label{s:beo_bes_automatsys}

Le but de l’automatisation du système est de rendre les mesures autonomes 24 heures sur 24, 7 jours sur 7.
Il doit pouvoir fonctionner pendant une période de service d’au moins 14 jours.
Il doit également être utilisable en zone éloignée notamment grâce à un système de génération d’alarmes.
Une caractéristique essentielle à son automatisation est le critère de configurabilité.
Le système doit pouvoir s’adapter à différents sites d’études (grand volume d’eau, petit volume d’eau, milieu sauvage, milieux de production, etc.).
% !TeX encoding = UTF-8
% !TeX spellcheck = fr_FR


\subsection{Communication à distance}
\label{s:beo_bes_commdist}

Le système doit assurer une communication à distance en tout temps.
Cet accès sans fil accordera une connexion à distance sécurisée et transfèrera des renseignements confidentiellement.
En cas de problème du système ou du logiciel, une alarme signalera le responsable du système afin de pouvoir régler ce problème et réparer le système.
De plus, une interaction à distance avec le capteur doit être assurée afin qu’il puisse fonctionner et être configuré de façon sans-fil, sans aller déplacer le système dans l’eau.
% !TeX encoding = UTF-8
% !TeX spellcheck = fr_FR


\subsection{Conservation des données}
\label{s:beo_bes_consdonn}
Il faut archiver les données prises par le capteur afin de pouvoir les valider ultérieurement.
Ces données doivent pouvoir être stockées pendant au moins deux ans.
Il faut enregistrer les images originales pour un traitement ultérieur à l’utilisation du capteur afin de valider les données.
% !TeX encoding = UTF-8
% !TeX spellcheck = fr_FR


\subsection{Autres considérations}
\label{s:beo_bes_autres}

Finalement, le client ordonne la livraison du projet final dans un délai raisonnable.
Il est également impératif que l’ensemble des coûts, des matériaux et de la main d’œuvre, demeurent à un niveau le bas possible.

% !TeX encoding = UTF-8
% !TeX spellcheck = fr_FR


\section{Objectifs}
\label{s:beo_obj}

% !TeX encoding = UTF-8
% !TeX spellcheck = fr_FR


\subsection{Optimisation de la prise de données}
\label{s:beo_obj_optimdonn}

Il faudra que notre modèle puisse analyser un volume d’intérêt d’au moins 1 m³ cube d’eau.
Il doit pouvoir identifier des poissons d’au moins 5 espèces différentes et d’au moins 6 cm de long.
Les conditions climatiques du Québec étant très variables, notre système doit tolérer une température externe entre 4 °C et 25 °C et une température interne entre -10 °C et 5 °C par rapport à la température de l’eau.
La masse totale de l’équipement submerger ne doit pas dépasser 5 kilogrammes pour un volume maximum de 0,3 mètre cube. Bien que léger, notre système devra être assez robuste pour fonctionner jusqu’à 15,25 m.
Les photos prises devront au minimum être codées en 8 bits dans une taille fixe de 100 pixels sur 100 pixels.
Chaque image devra inclure : l’heure, la date, la température interne du système, la température de l’eau et l’identification du poisson.
	% !TeX encoding = UTF-8
% !TeX spellcheck = fr_FR


\chapter{Cahier des charges}
\label{s:cdc}

Ce chapitre explique la procédure utilisée dans le but d'établir le cahier des charges.
Il expose les barèmes des différents objectifs.
De plus, le cahier des charges est synthétisé dans le tableau \ref{t:cdc_tab} et dans la maison de la qualité à la figure \textbf{LIEN VERS LA FIGURE}.

% !TeX encoding = UTF-8
% !TeX spellcheck = fr_FR


\newpage

\begin{table}[htp]
	\caption{Tableau du cahier des charges}
	\label{t:cdc_tab}
	\centering
	\begin{tabular}{|l|c|c|c|c|}
		\hline\hline
		\textbf{\textit{Critères d’évaluation}} & \textbf{\textit{Pond. (\%)}} & \textbf{\textit{Barème}} & \textbf{\textit{Min}} & \textbf{\textit{Max}} \\
		\hline
		\hline
		\ref{s:cdc_autonm} \textbf{Automatisation} & \textbf{20} & & & \\
		\ref{s:cdc_aut_autnmi} Autonomie & 7 & Équation \ref{eq:cdc_aut_autnmi} & & \\
		\ref{s:cdc_aut_basedonn} Base de données & 3 & & & \\
		\ref{s:cdc_aut_config} Configurabilité & 7 & & 0 & 100 \\
		\ref{s:cdc_aut_fiabsys} Fiabilité du système & 3 & & & \\
		\hline
		\hline
		\ref{s:cdc_comm} \textbf{Communication} & \textbf{15} & & & \\
		\ref{s:cdc_com_accadist} Accès à distance & 10 & Équation \ref{eq:cdc_com_accadist} & & \\
		\ref{s:cdc_com_faciutil} Facilité d’utilisation & 5 & & & \\
		\hline
		\hline
		\ref{s:cdc_coutsech} \textbf{Coûts et échéances} & \textbf{15} & & & \\
		\ref{s:cdc_cee_coutproj} Coût du projet & 10 & Tableau \ref{tab:cdc_cee_coutproj} & & \\
		\ref{s:cdc_cee_rspech} Facilité d’utilisation & 5 & Tableau \ref{tab:cdc_cee_rspech} & & \\
		\hline
		\hline
		\ref{s:cdc_prisdonn} \textbf{Prise de données} & \textbf{10} & & & \\
		\ref{s:cdc_pdd_cntphy} Contrainte physique & 5 & Tableau \ref{tab:cdc_pdd_cntphy} & & \\
		\ref{s:cdc_pdd_capt} Capteur & 5 & & & \\
		\hline
		\hline
		\ref{s:cdc_secu} \textbf{Sécurité} & \textbf{15} & & & \\
		\ref{s:cdc_sec_alar} Alarmes & 5 & & & \\
		\ref{s:cdc_sec_protaccdist} Protection de l’accès à distance & 4 & & & \\
		\ref{s:cdc_sec_protdonnclnt} Protection des données du client & 4 & & & \\
		\ref{s:cdc_sec_protenv} Protection de l’environnement & 2 & & & \\
		\hline
		\hline
	\end{tabular}
\end{table}
	% !TeX encoding = UTF-8
% !TeX spellcheck = fr_FR


\chapter{Conceptualisation et analyse de faisabilité}
\label{s:concpt_anals}


% !TeX encoding = UTF-8
% !TeX spellcheck = fr_FR


\definecolor{couleurbackintr}{HTML}{7F00FF}
\definecolor{couleurtextintr}{HTML}{FFFFFF}
\definecolor{couleurbackfonc}{HTML}{C780FF}
\definecolor{couleurtext}{HTML}{000000}
\definecolor{couleurbackextr}{HTML}{DDB3FF}

\newcommand{\multiliens}[2]{\foreach \noeud in {#1} {\draw[<-] (#2.west) -| ++(-1em,0em) |- (\noeud.east);}}
\newcommand{\lienhorz}[2]{\draw[->] (#1.east) -- (#2.west);}
\newcommand{\lienvtht}[2]{\draw[->] (#1.north) -- (#2.south);}
\newcommand{\lienvtbs}[2]{\draw[->] (#1.south) -- (#2.north);}


\begin{figure}[htp]
	\centering
	\tikzset{
		basic/.style={draw, rounded corners=2pt, thick, text width=8em, align=flush center, text=couleurtext, node distance=2em},
		intrant/.style={basic, fill=couleurbackintr, text=couleurtextintr},
		fonction/.style={basic, fill=couleurbackfonc},
		extrant/.style={basic, fill=couleurbackextr}
	}
	\begin{tikzpicture}[]
		\fontsize{10}{9} \selectfont
		% intrants
		\matrix[row sep=2em, column sep=2em] {
			% 1re ligne
			% TODO changer nom
			\node[intrant](forcutil){Force utilisée}; & \node[fonction](instcapt){Installer capteur sous l'eau}; & & & \node[extrant](captinst){Capteur installé}; \\
			% 2me ligne
			\node[intrant](defailla){Défaillance}; & \node[fonction](detcdefa){Détecter défaillance}; & & \node[fonction](genralrm){Générer alarme}; & \node[extrant](alarme){Alarme}; \\
			% 3me ligne
			\node[intrant](poissons){Poissons}; & \node[fonction](detecteu){Détecteur}; & \node[fonction](photo){Photo}; & \node[fonction](accudonn){Accumuler données}; & \\
			% 4me ligne
			\node[intrant](elecbatt){Électicité batterie}; & \node[fonction](alimcapt){Alimenter capteur}; & & \node[fonction](idenpois){Identifier poisson}; & \\
			% 5me ligne
			\node[intrant](tempreau){Température de l'eau}; & \node[fonction](captteau){Capteur température eau}; & & \node[fonction](archdonn){Archiver donnéees}; & \node[extrant](donnarch){Données archivées}; \\
			% 6me ligne
			\node[intrant](tempintr){Température interne}; & \node[fonction](capttcpt){Capteur température interne}; & & & \\
			% 7me ligne
			\node[intrant](confvolm){Configuration volume}; & \node[fonction](confcapt){Configurer capteur}; & & & \\
			% 8me ligne
			\node[intrant](entrutil){Entrées utilisateurs}; & \node[fonction](authutil){Authentifier utilisateur}; & \node[fonction](accedonn){Accéder aux données}; & & \\
		};
		
		% Liens
		\lienhorz{forcutil}{instcapt}
		\lienhorz{defailla}{detcdefa}
		\lienhorz{poissons}{detecteu}
		\lienhorz{elecbatt}{alimcapt}
		\lienhorz{tempreau}{captteau}
		\lienhorz{tempintr}{capttcpt}
		\lienhorz{confvolm}{confcapt}
		\lienhorz{entrutil}{authutil}
		
		\lienhorz{detcdefa}{genralrm}
		\lienhorz{instcapt}{captinst}
		
		\lienvtht{authutil}{confcapt}
	\end{tikzpicture}
	\caption{Diagramme fonctionnel}
	\label{f:caf_diag_fonc}
\end{figure}
% !TeX encoding = UTF-8
% !TeX spellcheck = fr_FR


\section{Archiver les données}
\label{s:faisab_archiver}
Le MFA exige que l’ensemble des données soient conservées pour un minimum de deux ans. Les données sont constituées des éléments suivants: images originales, vignettes, identification, date et heure, température du système, température de l’eau et alarmes. La capacité en mémoire et la fiabilité doivent donc être suffisants pour garantir l’intégrité des données sur 2 ans.

Aspects physiques:
-Le concept doit avoir assez de mémoire pour conserver 2 années d'information: 1 To.
-Le concept doit être assez fiable pour garantir l’intégrité des données sur 2 ans.
Aspects économiques:
-Le concept doit minimiser le coût 
Aspects temporels:
-Le concept doit pouvoir être construit et déployé dans les délais du projets.
Aspects sociaux-environnementaux:
-Non-applicable


% !TeX encoding = UTF-8
% !TeX spellcheck = fr_FR


\subsection{Serveur PowerEdge T30}
\label{s:archiver_conc1}

\textbf{Description}:La compagnie Dell offre un serveur prêt à accepter jusqu’à 4 disques durs pour 604,80\$[refDell]. Ce serveur sera installé au MFA. Le montant inclut une garantie de 2 ans sur le matériel avec livraison le jour suivant. En ajoutant 2 disques durs Western Digital Red de 1 To pour 149,98\$[refAmz] branchés en RAID1 [refRAID], on a 1 To de mémoire redondante. Ces disques ont un temps moyen entre pannes de 1000000 h, c-à-d 114 années.  Le coût total du matériel se limite à 754,68\$. 
\textbf{Décision}: Retenu mais 
\textbf{Justification}: Ce matériel permet d’entreposer une quantité suffisante de données. Le matériel du serveur est garanti sur 2 ans. Les disques durs de 1 To ont une cote de fiabilité dépassant largement 2 ans, en plus d’être installés de manière redondante. Les données n’ont toutefois aucune redondance géographique, ce qui augmente le risque de perte de données. Toutes les composantes sont disponibles immédiatement. Le coût total est acceptable dans le contexte du projet.
\textbf{Références}: 
https://www.dell.com/en-ca/work/shop/cty/pdp/spd/poweredge-t30/pe\_t30\_12084
579,00\$ (03-17, dispo maintenant)
25,80\$ pour garantie pièces sur 2 ans, next-day deliv
https://www.amazon.ca/dp/B008JJLXO6/ref=twister\_B008VQ8IKY?\_encoding=UTF8\&psc=1
74,99\$ (03-17, dispo maintenant)
WD RED spec sheet
https://en.wikipedia.org/wiki/Standard\_RAID\_levels#RAID\_1

% !TeX encoding = UTF-8
% !TeX spellcheck = fr_FR


\subsection{Serveur PowerEdge T30}
\label{s:archiver_conc1}

\textbf{Description}:La compagnie Dell offre un serveur prêt à accepter jusqu’à 4 disques durs pour 604,80\$[refDell]. Ce serveur sera installé au MFA. Le montant inclut une garantie de 2 ans sur le matériel avec livraison le jour suivant. En ajoutant 2 disques durs Western Digital Red de 1 To pour 149,98\$[refAmz] branchés en RAID1 [refRAID], on a 1 To de mémoire redondante. Ces disques ont un temps moyen entre pannes de 1000000 h, c-à-d 114 années.  Le coût total du matériel se limite à 754,68\$. 
\textbf{Décision}: Retenu mais 
\textbf{Justification}: Ce matériel permet d’entreposer une quantité suffisante de données. Le matériel du serveur est garanti sur 2 ans. Les disques durs de 1 To ont une cote de fiabilité dépassant largement 2 ans, en plus d’être installés de manière redondante. Les données n’ont toutefois aucune redondance géographique, ce qui augmente le risque de perte de données. Toutes les composantes sont disponibles immédiatement. Le coût total est acceptable dans le contexte du projet.
\textbf{Références}: 
https://www.dell.com/en-ca/work/shop/cty/pdp/spd/poweredge-t30/pe\_t30\_12084
579,00\$ (03-17, dispo maintenant)
25,80\$ pour garantie pièces sur 2 ans, next-day deliv
https://www.amazon.ca/dp/B008JJLXO6/ref=twister\_B008VQ8IKY?\_encoding=UTF8\&psc=1
74,99\$ (03-17, dispo maintenant)
WD RED spec sheet
https://en.wikipedia.org/wiki/Standard\_RAID\_levels#RAID\_1

% !TeX encoding = UTF-8
% !TeX spellcheck = fr_FR


\subsection{Serveur PowerEdge T30}
\label{s:archiver_conc1}

\textbf{Description}:La compagnie Dell offre un serveur prêt à accepter jusqu’à 4 disques durs pour 604,80\$[refDell]. Ce serveur sera installé au MFA. Le montant inclut une garantie de 2 ans sur le matériel avec livraison le jour suivant. En ajoutant 2 disques durs Western Digital Red de 1 To pour 149,98\$[refAmz] branchés en RAID1 [refRAID], on a 1 To de mémoire redondante. Ces disques ont un temps moyen entre pannes de 1000000 h, c-à-d 114 années.  Le coût total du matériel se limite à 754,68\$. 
\textbf{Décision}: Retenu mais 
\textbf{Justification}: Ce matériel permet d’entreposer une quantité suffisante de données. Le matériel du serveur est garanti sur 2 ans. Les disques durs de 1 To ont une cote de fiabilité dépassant largement 2 ans, en plus d’être installés de manière redondante. Les données n’ont toutefois aucune redondance géographique, ce qui augmente le risque de perte de données. Toutes les composantes sont disponibles immédiatement. Le coût total est acceptable dans le contexte du projet.
\textbf{Références}: 
https://www.dell.com/en-ca/work/shop/cty/pdp/spd/poweredge-t30/pe\_t30\_12084
579,00\$ (03-17, dispo maintenant)
25,80\$ pour garantie pièces sur 2 ans, next-day deliv
https://www.amazon.ca/dp/B008JJLXO6/ref=twister\_B008VQ8IKY?\_encoding=UTF8\&psc=1
74,99\$ (03-17, dispo maintenant)
WD RED spec sheet
https://en.wikipedia.org/wiki/Standard\_RAID\_levels#RAID\_1

% !TeX encoding = UTF-8
% !TeX spellcheck = fr_FR


\subsection{Serveur PowerEdge T30}
\label{s:archiver_conc1}

\textbf{Description}:La compagnie Dell offre un serveur prêt à accepter jusqu’à 4 disques durs pour 604,80\$[refDell]. Ce serveur sera installé au MFA. Le montant inclut une garantie de 2 ans sur le matériel avec livraison le jour suivant. En ajoutant 2 disques durs Western Digital Red de 1 To pour 149,98\$[refAmz] branchés en RAID1 [refRAID], on a 1 To de mémoire redondante. Ces disques ont un temps moyen entre pannes de 1000000 h, c-à-d 114 années.  Le coût total du matériel se limite à 754,68\$. 
\textbf{Décision}: Retenu mais 
\textbf{Justification}: Ce matériel permet d’entreposer une quantité suffisante de données. Le matériel du serveur est garanti sur 2 ans. Les disques durs de 1 To ont une cote de fiabilité dépassant largement 2 ans, en plus d’être installés de manière redondante. Les données n’ont toutefois aucune redondance géographique, ce qui augmente le risque de perte de données. Toutes les composantes sont disponibles immédiatement. Le coût total est acceptable dans le contexte du projet.
\textbf{Références}: 
https://www.dell.com/en-ca/work/shop/cty/pdp/spd/poweredge-t30/pe\_t30\_12084
579,00\$ (03-17, dispo maintenant)
25,80\$ pour garantie pièces sur 2 ans, next-day deliv
https://www.amazon.ca/dp/B008JJLXO6/ref=twister\_B008VQ8IKY?\_encoding=UTF8\&psc=1
74,99\$ (03-17, dispo maintenant)
WD RED spec sheet
https://en.wikipedia.org/wiki/Standard\_RAID\_levels#RAID\_1

% !TeX encoding = UTF-8
% !TeX spellcheck = fr_FR


\begin{table}[!htbp]
	\begin{tabular}{|l|c|c|c|c|c|}
		\hline
		\multicolumn{1}{|c|}{\multirow{2}{*}{\textbf{Concepts}}} & \multicolumn{4}{c|}{\textbf{Aspects de l'analyse}} & \multirow{2}{*}{\textbf{Décision}} \\ \cline{2-5}
		\multicolumn{1}{|c|}{}                                   & Physiques & Économiques & Temporels & Socio-envir. &                                    \\ \hline
		PowerEdge T30                                                 & OUI MAIS       & OUI         & OUI       & OUI          & RETENU MAIS                             \\ \hline
		Nuage Amazon S3                                                 & OUI       & OUI         & OUI       & OUI          & RETENU                             \\ \hline
		Nuage MS Azure                                                 & OUI       & OUI         & OUI       & OUI          & RETENU                             \\ \hline
		Serveur OVH                                                & OUI MAIS      & OUI MAIS        & OUI       & OUI          & RETENU MAIS        \\ \hline
	\end{tabular}
	\caption{Décisions pour fonction "archiver les données"}
	\label{tab:fct_archiver}
\end{table}

% !TeX encoding = UTF-8
% !TeX spellcheck = fr_FR


\section{Identifier les espèces de poissons}
\label{s:faisab_identifier}
Le MFA exige pouvoir identifier 5 espèces de poissons identifiables par site étudié. L’identification doit être faite de manière autonome partir d’une vignette 100 par 100 pixels. Les concepts retenus doivent non seulement identifier les espèces mais offrir niveau de fiabilité acceptable. L’apprentissage automatique (de l’anglais “machine learning”) semble être la technologie à favoriser dans notre cas. Le choix de plateforme pour l’apprentissage automatique devrait minimiser le coût et le temps de développement.

Puisqu’il n’existe pas de banque d’image permettant d’entraîner la machine pour cette tâche, une telle collection devra être créée pour le projet. Environ 200 images identifiées par espèce devraient suffir pour entraîner l’algorithme.\cite{deeplearn_2015, fishID_2016} Faire classifier 200 images par un biologiste devrait prendre au plus 1 journée, pour un total de 240\$ \cite{sal_biologiste} par espèce. Les images à identifier manuellement proviendront des premières images générées par le projet. Accumuler 200 images d’une nouvelle espèce pourrait prendre jusqu’à 1 mois, mais s’effectue en parallèlement pour toutes les espèces d’un milieu particulier.

Aspects physiques:
\begin{itemize}
	\item L’identification doit pouvoir être fiable
	\item Doit pouvoir identifier au minimum 5 espèces de poisson.
\end{itemize}

Aspects économiques:
\begin{itemize}
	\item Limiter le coût de développement.
	\item Limiter le coût d’achat de matériel, si applicable.
\end{itemize}

Aspects temporels:
\begin{itemize}
	\item Le concept doit pouvoir être développé dans les délais du projet.
\end{itemize}

Aspects sociaux-environnementaux:
\begin{itemize}
	\item Non-applicable
\end{itemize}


% !TeX encoding = UTF-8
% !TeX spellcheck = fr_FR


\subsection{TensorFlow sur Google Cloud}
\label{s:identifier_conc1}

\textbf{Description}:TensorFlow est la plateforme “open source” d'apprentissage automatique de Google. L’utilisation de TensorFlow est gratuite en respectant la licence. Avec une banque d’image pré-identifiées, 3 semaines de travail sont estimée pour développer l'algorithme d’apprentissage et l’entraîner. Ceci coûtera environ 4615\$, selon le salaire moyen d’un ingénieur logiciel[ref:salingLog]. Dans ces conditions la fiabilité de l’identification devrait approcher 95\%. Googl Cloud est un service nuagique qui permet l’apprentissage automatique. Faire fonctionner cet algorithme sur Google Cloud pendant 2 ans devrait coûter aux environs de 2200\$. Développer un banque d’images pour 5 espèces monte à 1200\$ et 1 semaine de travail. Incluant la banque d’image, ce concept nécessitera donc un total de 4 semaines de travail, 2200\$ en matériel et 5815\$.
\textbf{Décision}: Retenu
\textbf{Justification}: Les coût de développement et de matériel sont en raisonnables. Les ressources choisies sont appropriées pour obtenir une identification fiable d’un nombre approprié d’espèces. Un mois de développement est acceptable dans le cadre du projet.
\textbf{Références}:
\cite{googCloud, tensorFlow, liscapache2,sal_ingLog, fishID_2016}
% !TeX encoding = UTF-8
% !TeX spellcheck = fr_FR


\subsection{Tensorflow sur Amazon SageMaker}
\label{s:identifier_conc2}

\textbf{Description}:Tel que mentionné plus haut [ref point1] TensorFlow est une plateforme d’apprentissage automatique libre d’utilisation.  Avec une banque d’image pré-identifiées, 3 semaines de travail sont estimées pour développer l'algorithme d’apprentissage et l’entraîner. Selon le salaire moyen d’un ingénieur logiciel, le coût est de 4615\$.[ref:salingLog].  La fiabilité de l’identification devrait approcher 95\%[ref:fishID_2016]. SageMaker est le service nuagique d’Amazon qui permet l’apprentissage automatique. Faire fonctionner cet algorithme sur SageMaker pendant 2 ans devrait coûter aux environs de 1900\$[ref:siteamznSage]. Développer un banque d’images pour 5 espèces monte à 1200\$ et 1 semaine de travail. Ce concept nécessitera donc un total de 4 semaines de travail, 1900\$ en matériel et 5815\$ en développement.
\textbf{Décision}: Retenu
\textbf{Justification}: Les coût de développement et de matériel sont en raisonnables. Les ressources choisies sont appropriées pour obtenir une identification fiable d’un nombre approprié d’espèces. Un mois de développement est acceptable dans le cadre du projet.
\textbf{Références}: tensorFlow, amzSage
% !TeX encoding = UTF-8
% !TeX spellcheck = fr_FR


\subsection{PyTorch sur Google Cloud}
\label{s:identifier_conc3}

\textbf{Description}:PyTorch est une autre plateforme d’apprentissage automatique libre d’utilisation.[ref:pyTorch]  Avec une banque d’image pré-identifiées, 3 semaines de travail sont estimée pour développer l'algorithme d’apprentissage et l’entraîner. Ceci coûtera environ 4615\$, selon le salaire moyen d’un ingénieur logiciel[ref:salingLog]. Dans ces conditions la fiabilité de l’identification devrait approcher 95\%[ref:fishID2016]. Faire fonctionner cet algorithme sur Google Cloud pendant 2 ans devrait coûter aux environs de 2200\$. Développer un banque d’images pour 5 espèces monte à 1200\$ et 1 semaine de travail. Incluant la banque d’image, ce concept nécessitera donc un total de 4 semaines de travail, 2200\$ en matériel et 5815\$.

\textbf{Décision}: Retenu

\textbf{Justification}: Les coût de développement et de matériel sont en raisonnables. Les ressources choisies sont appropriées pour obtenir une identification fiable d’un nombre approprié d’espèces. Un mois de développement est acceptable dans le cadre du projet.

\textbf{Références}:
\cite{pytorch, googCloud, fishID_2016, sal_ingLog} 	
% !TeX encoding = UTF-8
% !TeX spellcheck = fr_FR


\subsection{Keras sur Google Cloud}
\label{s:identifier_conc4}

\textbf{Description}: Keras est une librairie libre d’usage qui facilite le développement de d’applications d’apprentissage automatique avec TensorFlow[ref:keras].  L’utilisation de TensorFlow est libre d’usage[ref:tensorFlow]. Avec une banque d’image pré-identifiées, 2 semaines de travail sont estimée pour développer l'algorithme d’apprentissage et l’entraîner. Ceci coûtera environ 3077\$, selon le salaire moyen d’un ingénieur logiciel[ref:sal_ingLog]. Dans ces conditions la fiabilité de l’identification devrait approcher 95\%[ref:fishID_2016]. Google Cloud est un service nuagique qui permet l’apprentissage automatique. Faire fonctionner cet algorithme sur Google Cloud pendant 2 ans devrait coûter aux environs de 2200\$. Développer un banque d’images pour 5 espèces monte à 1200\$ et 1 semaine de travail. Incluant la banque d’image, ce concept nécessitera donc un total de 4 semaines de travail, 2200\$ en matériel et 4276\$.
\textbf{Décision}: Retenu
\textbf{Justification}: Les coût de développement et de matériel sont en raisonnables. Les ressources choisies sont appropriées pour obtenir une identification fiable d’un nombre approprié d’espèces. Un mois de développement est acceptable dans le cadre du projet.
\textbf{Références}: keras, googCloud, tensorFlow, fishID_2016
% !TeX encoding = UTF-8
% !TeX spellcheck = fr_FR



\begin{table}[!htbp]
	\begin{tabular}{|l|c|c|c|c|c|}
		\hline
		\multicolumn{1}{|c|}{\multirow{2}{*}{\textbf{Concepts}}} & \multicolumn{4}{c|}{\textbf{Aspects de l'analyse}} & \multirow{2}{*}{\textbf{Décision}} \\ \cline{2-5}
		\multicolumn{1}{|c|}{}                                   & Physiques & Économiques & Temporels & Socio-envir. &                                    \\ \hline
		TensorFlow Google Cloud                                                 & OUI       & OUI         & OUI       & OUI          & RETENU                             \\ \hline
		Tensorflow SageMaker                                                 & OUI       & OUI         & OUI       & OUI          & RETENU                             \\ \hline
		PyTorch Google Cloud                                                 & OUI       & OUI         & OUI       & OUI          & RETENU                             \\ \hline
		Keras Google Cloud                                                 & OUI       & OUI         & OUI       & OUI          & RETENU	        \\ \hline
	\end{tabular}
	\caption{Décisions pour fonction "identifier les poissons"}
	\label{tab:fct_identifier}
\end{table}


	% !TeX encoding = UTF-8
% !TeX spellcheck = fr_FR


\addcontentsline{toc}{chapter}{\bibname}%
\renewcommand{\bibname}{Bibliographie}
\def\UrlBreaks{\do\/\do-}
\Urlmuskip=0mu plus 1mu
\bibliographystyle{ieeetr}
\bibliography{res/biblio}

	
	%   Annexes
	\appendix
	% !TeX encoding = UTF-8
% !TeX spellcheck = fr_FR

%
% Annexe "Liste des sigles et des acronymes"
%


\chapter{Liste des sigles et des acronymes}

% Ne pas y inclure les unités SI

\begin{flushleft}
   \begin{tabular}{@{}ll}
%   	Format:
%		FA			& Fabuleux acronyme									\\
		MFA			& Ministère de la Faune aquatique					\\
   \end{tabular}
\end{flushleft}

	
\end{document}
% Fin du document
